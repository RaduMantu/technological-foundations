\section{Overview}

Data processing and visualization are fundamental components of computer science
that enable the transformation of raw data into meaningful insights. Data
processing involves collecting, cleaning, organizing, and analyzing data, often
using algorithms and software tools to uncover patterns and support decision
making. Visualization complements this by representing data graphically,
allowing users to more easily comprehend complex datasets through charts,
graphs, and other visual formats. Together, these techniques play a critical
role in fields such as data science, artificial intelligence, and software
engineering.

From a technical standpoint, data processing encompasses a wide range of
operations, such as data transformation, aggregation, filtering, and statistical
analysis. These are often achieved using programming languages like Python,
R, Matlab, etc. Note that scripting languages are usually preferable to their
compiled counterparts due to their flexibility that allows quick prototyping.
However, these scripts act as glue code for the underlying modules or libraries
that implement the data processing algorithms. These algorithms are usually
written in compiled languages such as C++ and then interfaced with the
interpretor of the script (e.g., \textbf{python3}) via what is called the
\textbf{Foreign Function Interface (FFI)}. As a rule, first check if your data
processing algorithm is al- ready implemented in your scripting language /
framework of choice. The only reason why you would want to implement your own in
that scripting language is if it presents a significant algorithmic improvement.
