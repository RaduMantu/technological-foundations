\newpage

\section{Practical exercises}

\subsection{Disk usage visualization}

The \href{https://man.archlinux.org/man/du.1}{du} command can be used to
estimate the disk utilization of a file, or recursively scan a directory. When
using this command, you need to note that it does not differentiate between
cohesive files and \href{https://wiki.archlinux.org/title/Sparse_file}{sparse
files}. Additionally, \href{https://linuxhandbook.com/hard-link/}{hard links}
can trick you into thinking that your disk utilization can exceed the actual
storage space.

These are a few caveats that you would normally need to keep in mind, but for
this task we are going to focus on the plotting aspect. Go in your home
directory and use \texttt{du} recursively (see the \texttt{-s} flag) on all
subdirectories, but also any files residing directly in your home. Output their
size in bytes instead of kilobytes, then use the
\href{https://man.archlinux.org/man/sort.1}{sort} command to sort the items in
reverse numerical order, based on size (meaning largest first).

Write a Python script that takes this as input and generates a
\href{https://pythonguides.com/matplotlib-plot-bar-chart/}{bar plot}
illustrating how much space each file and subdirectory in your home takes up.

\subsection{CPU usage visualization}

As you know, your CPU has multiple cores. To find out exactly how many, you can
run the \href{https://man.archlinux.org/man/nproc.1}{nproc} command. In this
task we are going to run CPU intensive workloads on \textit{specific} cores and
create a bar plot representing the percentage of time each core spent executing
user level code (i.e., applications), system level code (i.e., kernel tasks
unrelated to servicing hardware peripherals) and hardware interrupts.

To gather this data, you can use the
\href{https://man.archlinux.org/man/mpstat.1}{mpstat} command. To
perform CPU-intensive work, take a look at the
\href{https://man.archlinux.org/man/stress.1}{stress} command. If left as it is,
the kernel's scheduler will most likely bounce the \texttt{stress} process from
one CPU core to another. Try using the
\href{https://man.archlinux.org/man/taskset.1}{taskset} command to bind multiple
\texttt{stress} workers to \textit{odd} CPU cores.

\subsection{Dynamic plotting}

Make sure that you have completed the previous task before starting this one.

\texttt{mpstat} can either report momentary statistics or it can do so
periodically. Try alternating the CPU load between even and odd cores on your
processor (see \href{https://man.archlinux.org/man/timeout.1}{timeout} and
\href{https://man.archlinux.org/man/sleep.1}{sleep}) and visualize these
statistics with a
\href{https://holypython.com/python-visualization-tutorial/creating-bar-chart-animations/}
{dynamic bar plot}.

\subsection{Pandas introduction}

Pandas is a Python library used in data manipulation and analysis. It is based
on \texttt{numpy} but introduces certain data structures such as
\href{https://pandas.pydata.org/pandas-docs/stable/reference/api/pandas.DataFrame.html}
{DataFrames} for managing tabular data, or
\href{https://pandas.pydata.org/pandas-docs/stable/reference/api/pandas.Series.html}
{Series} for handling 1D arrays (especially time series). Pandas also offers
tools for interacting with different storage formats such as CSV, Excel, SQL
databases, etc.

We recommend going over the official
\href{https://pandas.pydata.org/pandas-docs/stable/user_guide/10min.html}{10
minutes to pandas} crash course and keeping this
\href{https://github.com/pandas-dev/pandas/blob/main/doc/cheatsheet/Pandas_Cheat_Sheet.pdf}
{cheat sheet} at hand. Note that Pandas also has built-in plotting support,
even though it's based on \texttt{matplotlib}. See the
\href{https://pandas.pydata.org/pandas-docs/stable/user_guide/visualization.html}
{chart visualization} chapter in the official documentation and also look at
the \href{https://pandas.pydata.org/pandas-docs/stable/user_guide/cookbook.html#cookbook-plotting}
{cookbook} chapter for practical examples.

Try downloading \href{https://www.kaggle.com/datasets/themlphdstudent/countries-population-from-1955-to-2020}
{this kaggle dataset} containing per-country population statistics between
1955 and 2020 and plotting the demographic changes that can be extracted from
this data. Start with your own country and then perform a comparison with
immediate neighbours.

\subsection{Radio spectrogram}

In the assets directory, we have a \texttt{iq\_sample.raw} file containing
measurements of the FM radio spectrum performed with a
\href{https://blinry.org/50-things-with-sdr/}{Software Defined Radio (SDR)}. In
this exercise we are going to process these samples and obtain a spectrogram,
showing how the radio signal frequency changes in time.

\begin{figure}[h]
    \centering
    \includegraphics[width=\textwidth,keepaspectratio]{figures/waterfall-1.png}
    \caption{Spectrogram of radio signal (different form our sample).}
    \label{fig:spectrogram}
\end{figure}

In Figure \ref{fig:spectrogram} we see what is called a waterfall plot. The
signal that we recorded is comprised of multiple other signals of \textit{specific}
frequencies. Each line of pixels represents the amount of power each individual
component exhibits in a limited time frame. This information is encoded through
the pixel color, that adheres to the
\href{https://www.mathworks.com/help/matlab/ref/jet.html}{jet colormap} (i.e.,
blue is weaker, red is stronger). In this plot we notice that there are a few
frequency bands that consistently present high levels of power. These are
different radio stations. The small shifts in power distribution around these
frequencies over time is how sound is encoded when transmitted to our radio
devices.

\textbf{Note:} The sample we are going to be working on is focused on a single
radio station. We intentionally reduced the scope in order to keep the sample
size (and the RAM consumption) within reasonable limits.

\subsubsection{Parsing input data}

In order to acquire this data, we used a
\href{https://www.nuand.com/product/bladerf-xa9/}{bladeRF 2.0 micro xA9} SDR
and \href{https://www.gqrx.dk/}{GQRX} for data visualization and acquisition on
our host. We tuned our SDR to a frequency of $101.3MHz$ and used a sampling rate
of $1MHz$. The sampling rate determines how wide the frequency spectrum that
we visualize can be. The Ox axis of our previous figure is defined between
$103.4 \pm 5MHz$. In our sample, the range will be $101.3 \pm 0.5MHz$.

SDRs usually work with complex sampling, also known as
\href{https://pysdr.org/content/sampling.html}{In-phase Quadrature (I/Q)
sampling}. Instead of providing a real value representing the amplitude of the
signal at a given time, this technique provides a number of benefits. These are
not relevant to the exercise at hand but just to get an idea, if we were working
with \textit{real} numbers instead of \textit{complex} numbers, we wouldn't be
able to \textit{"see"} frequencies at $\pm500kHz$ around our tuned radio
frequency. Instead, we would only be able to see the positive half of the spectrum.

GQRX saves raw I/Q data as \textbf{32-bit floating-point complex samples}. The
first four bytes represent the real component and the last two bytes represent
the imaginary part. The exact values do not necessarily interest us since we
only want to observe differences between the presence of a signal and a lack
thereof. Information about the range of measurable values can be found in the
data sheet of the SDR.

To start things off, read the data from the provided \textbf{binary file} and
use the \texttt{struct} module to \textbf{unpack} the data into floating point
numbers. Create an array of \textbf{complex} values representing our samples.
Note that Python has built-in support for complex number representation (e.g.,
\texttt{1 + 2j}).

\vskip .5cm

\textbf{Tip \#1:} If you notice that you are running low on RAM while solving
this exercise, consider employing
\href{https://dspguru.com/dsp/faqs/multirate/decimation/}{signal decimation}.
Effectively, you can select only every second, or fourth, or eighth, etc.
sample. This will reduce the memory consumption but the width of the frequency
spectrum will also be reduced by a factor of 2, 4, 8, and so on. To track the
available memory, use the following command (in a different terminal). Avoid
letting the \textbf{Mem - free} value drop to zero, even if you have
\href{https://wiki.archlinux.org/title/Swap}{Swap Memory}.

\begin{lstlisting}[style=bashstyle]
# invoke the free command every second
$ watch -n 1 free -h
\end{lstlisting}

\textbf{Tip \#2:} When you are done with certain data stored in a variable
\textbf{x} let's say, consider running \textbf{del(x)} to free up the memory.
For this subtask, you may load the raw I/Q sample file in memory and then
process it into an array of complex numbers. If the array is stored in a
different variable than the raw data, the garbage collector will not free up
the raw data buffer since there is still a reference to it.

\subsubsection{Convert to frequency domain}

Now that we have an array of I/Q samples, we can convert the signal from the
time domain to the frequency domain. This can be done using the
\href{https://numpy.org/doc/2.1/reference/generated/numpy.fft.fft.html#numpy.fft.fft}
{Fast Fourier Transform (FFT)}. The FFT takes a time series of complex numbers
and outputs yet another series of complex numbers, but this new series will
be in the frequency domain. The range of frequencies is determined
\textit{solely} by the tuned frequency ($101.3MHz$) and the sampling frequency
($1MHz$ if not decimated).

At this point, we could compute the FFT on the entirety of our sampled signal.
However, this will only show us the amount of power dissipated for each frequency
during the entire lifetime of the signal (or rather, our measurement). In other
words, we would not be able to see the \textit{evolution in time}. The solution
is simple. We must create a \textit{sliding window}; a continuous subset of
samples on which we apply the FFT. By shifting the window to the right, we
advance in time but maintain a limited scope for our frequency analysis. The
result of this operation will be an array of FFT outputs, each representing a
\textbf{line} in our final plot.

Note that the first element of the FFT's output corresponds to the central
frequency ($F_t$) of $101.3MHz$. For an output of size $N$, the first $N / 2$
elements correspond to the $[ F_t : F_t + F_s / 2 )$ range. Half of the sampling
frequency ($F_s$) is also known as the
\href{https://en.wikipedia.org/wiki/Nyquist_frequency}{Nyquist frequency}. The
latter $N / 2$ elements of the FFT output wrap around the range of frequencies
and represent $[ F_t - F_s / 2 : F_t )$. As a result, each FFT output must be
rotated by $N / 2$ elements. This is such a common operation that there is even
a \texttt{numpy} function called
\href{https://numpy.org/doc/stable/reference/generated/numpy.fft.fftshift.html}
{fftshift}. Makes sure to apply it to \textit{every} FFT output you obtain.

\vskip .5cm

\textbf{Tip \#1:} Try to pick a power of 2 as the size of the window. This helps
optimize the computation of the FFT.

\textbf{Tip \#2:} It is your choice by how much you wish to advance the sliding
window. Yes, you can have overlaps between consecutive snapshots. Just note that
the smaller you step is, the longer it will take to finish the computation.
More lines, means more FFT function calls (which are expensive), means more time
spent prototyping. Consider using a step value equal to the size of the window.

\textbf{Tip \#3:} When choosing the FFT window size, remember that this will
\textbf{not} affect the frequency visibility range. It will still be $101.3 \pm
0.5MHz$ regardless of your choice. What will be affected however is the precision.
If you compute a FFT on a small number of samples, let's say 100, then the range
of frequencies that is determined by the sampling frequency (i.e., $1MHz$) will
be split into 100 "buckets" of $10kHz$ each. If two or more \textit{distinct}
signals fall withing the same "bucket", they will be represented as a single
signal. If you decide to use 1,000 samples instead, you will have $1kHz$ buckets
instead of $10kHz$ buckets, which may be enough for you to distinguish the two
signals.

\subsubsection{Post processing}

Now that we have one complex value for each pixel that we should render, the
question is how do we visualize this data? First of all, we need to convert each
complex number to a real domain value by computing its magnitude. Yes, there is
a function for this in the \texttt{numpy} module.

While computing the magnitudes, our suggestion is to also compute the natural
logarithm of these values. In signal processing the decibel (dB) scale is
commonly used in order to represent relative differences in power or amplitude,
which are more significant than absolute differences. This is also why we did
not bother to look up what the I and Q component value range is, at the start of
this exercise. Additionally, using a logarithmic scale can help emphasize lower
frequencies. This is especially helpful in speech recognition but it also
severs to highlight certain features in this example.

\vskip .5cm

\textbf{Tip \#1:} Note that the result may be noisy. In this case, you could
perform a \href{https://en.wikipedia.org/wiki/Moving_average}{moving average}
over each line to smooth out the data. The process is similar to the sliding
window FFT computation, but with a few notable distinctions. First, we are going
to be using the
\href{https://numpy.org/doc/stable/reference/generated/numpy.average.html}{
average} function instead of the FFT. Second, when computing a moving average
it is best to have a small step relative to the window size. Having a high
overlap between consecutive windows helps maintain the cohesion of the data that
we want to display. Finally, the window size itself should be as small as
possible. If it's too large, we will eliminate relevant features from our data.
However, if it's too small, the result will still contain noise. For a first
attempt, try to generate the image without any averaging and decide for yourself
if it is even necessary.

\subsubsection{Plotting}

Finally, use the
\href{https://matplotlib.org/stable/api/_as_gen/matplotlib.pyplot.imshow.html}{
imshow} function from \texttt{pyplot} to plot the data. The \texttt{cmap}
argument can be used to select the color grading for the heatmap. In our example
use used \texttt{jet} but there are
\href{https://matplotlib.org/stable/users/explain/colors/colormaps.html} many
alternatives. Also, use the \texttt{extent} argument to set the numeric ranges
for the Ox and Oy axes. We've already mentioned what the frequency range is.
The elapsed time for the signal measurement can be computed based on the number
of samples, the sampling frequency, and the size of each sample (in bytes).

The result should look something like this:

\begin{figure}[h]
    \centering
    \includegraphics[width=\textwidth,keepaspectratio]{figures/waterfall-2.png}
    \caption{Example output for the $F_s = 1MHz$ sample. Duration limited to $2s$.}
    \label{fig:spectrogram}
\end{figure}

