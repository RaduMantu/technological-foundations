\section{Overview}

Data storage is a critical aspect of managing information in any system,
ensuring that data is securely saved, easily accessible, and efficiently
organized. While local file systems or
\href{https://cloud.google.com/learn/what-is-object-storage}{object storage} are
used for simpler applications, cloud-based storage systems (e.g., Amazon S3,
Google Cloud Storage, etc.) can provide a more scalable and durable alternative
albeit at a higher cost. The choice of storage depends on many factors (e.g.,
data type, access patterns, scalability needs, cost, etc.) and should be made on
a case by case basis. Ensuring data integrity, security, and availability
through redundancy and backups is paramount in any storage strategy.

Versioning data is the process of tracking and managing changes to data over
time, allowing users to access previous states, compare changes, or revert to
earlier versions if needed. This is particularly important in collaborative
environments or systems where data evolves, such as software development,
machine learning datasets, or content management. Version control systems like
\href{https://github.com/git-guides}{Git} are commonly used for code and
text-based data, storing snapshots of changes in a repository with metadata like
timestamps and authorship. Although Git is more often than not used to also
store non-text-based files (e.g., PDFs, binary blobs, etc.), alternatives such
as Data Version Control (DVC) or Delta Lake provide better versioning
capabilities for large data sets or binary data types. Keep in mind that certain
aspects of version control such as collaborative editing of documents can be
tied into a product ecosystem. For example, editing docx or xlsx files from the
Microsoft Office suite can be collaborative insofar as using OneDrive or
SharePoint as storage environments. Free, open-source alternatives exist but
those require that you switch to a different software stack that can integrate
with third-party solutions.

Effective data storage and versioning require careful planning to balance
performance, cost, and usability. Storage systems should be optimized for
read/write speeds and scalability, while versioning systems must handle
conflicts and provide clear change histories.

