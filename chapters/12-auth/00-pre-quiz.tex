\section{Pre-lecture Quiz}
\label{sec:sec:pre-quiz}

\subsection{Quiz Questions}

\textbf{Question 1:} What are the three main security objectives that a secure system should achieve?

\begin{enumerate}
    \item[A)] Authentication, Authorization, Access Control
    \item[B)] \textbf{Confidentiality, Integrity, Availability}
    \item[C)] Encryption, Hashing, Digital Signatures
    \item[D)] Prevention, Detection, Response
\end{enumerate}

\textbf{Question 2:} In the context of computer security, what is the difference between a bug and a vulnerability?

\begin{enumerate}
    \item[A)] A bug is always more serious than a vulnerability
    \item[B)] A vulnerability is a bug that can be exploited by an attacker
    \item[C)] \textbf{A bug becomes a vulnerability when it can be exploited by an attacker for their benefit}
    \item[D)] There is no difference between a bug and a vulnerability
\end{enumerate}

\textbf{Question 3:} What is the main advantage of asymmetric encryption (public key cryptography) over symmetric encryption?

\begin{enumerate}
    \item[A)] It is faster than symmetric encryption
    \item[B)] It provides better security than symmetric encryption
    \item[C)] \textbf{There is no need to transmit the encryption key through the network}
    \item[D)] It uses shorter keys than symmetric encryption
\end{enumerate}

\textbf{Question 4:} In the subject-object model of access control, what is the role of the reference monitor?

\begin{enumerate}
    \item It generates authentication tokens for users
    \item \textbf{It allows or denies subjects' access to objects based on configured permissions}
    \item[C)] It encrypts data before storage
    \item[D)] It monitors system performance
\end{enumerate}

\textbf{Question 5:} What is the principle of least privilege in computer security?

\begin{enumerate}
    \item[A)] Users should have the maximum possible access to system resources
    \item[B)] \textbf{Components should have access only to the minimum necessary resources to function}
    \item[C)] Only administrators should have access to system resources
    \item[D)] All users should have equal access to all resources
\end{enumerate}

\subsection{Answer Key}

\begin{enumerate}
    \item \textbf{B} - The three main security objectives are Confidentiality (ensuring only authorized users can access data), Integrity (ensuring data is not modified without authorization), and Availability (ensuring the system provides services consistently).
    
    \item \textbf{C} - A bug is an unexpected situation that leads to inadequate system functioning. A bug becomes a vulnerability when it can be exploited by an attacker for their benefit.
    
    \item \textbf{C} - Asymmetric encryption uses a public-private key pair where the public key can be shared openly, eliminating the need to securely transmit encryption keys through the network, which is a major challenge in symmetric encryption.
    
    \item \textbf{B} - The reference monitor is a privileged entity (usually the operating system) that enforces access control by allowing or denying subjects' access to objects based on configured permissions in the system.
    
    \item \textbf{B} - The principle of least privilege states that components (processes, users, applications) should have access only to the minimum necessary resources required for their proper functioning, limiting potential damage if compromised.
\end{enumerate}
