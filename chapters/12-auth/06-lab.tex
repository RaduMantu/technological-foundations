\newpage

\section{Practical exercises}

\subsection{User and Permission Management}

In this exercise, you will practice working with file permissions and understanding access control concepts.

First, create a test environment:

\begin{lstlisting}[style=bashstyle]
# Create test files and directories
$ mkdir -p ~/security-lab/{shared,private,group-only}
$ touch ~/security-lab/shared/public-file.txt
$ touch ~/security-lab/private/secret-file.txt
$ touch ~/security-lab/group-only/team-file.txt

# Set different permission levels
$ chmod 755 ~/security-lab/shared
$ chmod 700 ~/security-lab/private
$ chmod 750 ~/security-lab/group-only

# Set file permissions
$ chmod 644 ~/security-lab/shared/public-file.txt
$ chmod 600 ~/security-lab/private/secret-file.txt
$ chmod 640 ~/security-lab/group-only/team-file.txt
\end{lstlisting}

Practice viewing and understanding permissions:

\begin{lstlisting}[style=bashstyle]
# View file permissions
$ ls -la ~/security-lab/
$ ls -la ~/security-lab/shared/
$ ls -la ~/security-lab/private/
$ ls -la ~/security-lab/group-only/

# Understand permission notation
$ stat ~/security-lab/shared/public-file.txt
$ stat ~/security-lab/private/secret-file.txt
\end{lstlisting}

\textbf{Exercise}: Practice permission management:
\begin{itemize}
    \item Use \cmd{chmod} to change file and directory permissions
    \item Use \cmd{ls -la} to view permission information
    \item Use \cmd{stat} to see detailed permission information
    \item Understand the difference between read, write, and execute permissions
    \item Practice with different permission combinations (755, 644, 600, etc.)
\end{itemize}

\subsection{SSH and Remote Access}

Practice working with SSH keys and understanding secure remote access concepts.

Generate SSH key pairs:

\begin{lstlisting}[style=bashstyle]
# Generate SSH key pair
$ ssh-keygen -t ed25519 -C "student@example.com"

# View public key
$ cat ~/.ssh/id_ed25519.pub

# View private key (be careful with this)
$ ls -la ~/.ssh/
\end{lstlisting}

Practice SSH configuration:

\begin{lstlisting}[style=bashstyle]
# Create SSH config file
$ mkdir -p ~/.ssh
$ cat > ~/.ssh/config << EOF
Host myserver
    HostName 192.168.1.100
    User student
    Port 22
    IdentityFile ~/.ssh/id_ed25519
EOF

# Set proper permissions
$ chmod 600 ~/.ssh/config
$ chmod 600 ~/.ssh/id_*
$ chmod 644 ~/.ssh/id_*.pub
\end{lstlisting}

\textbf{Exercise}: Practice SSH concepts:
\begin{itemize}
    \item Use \cmd{ssh-keygen} to generate SSH key pairs
    \item Understand the difference between public and private keys
    \item Use \cmd{chmod} to set proper permissions on SSH files
    \item Understand SSH configuration file format
    \item Practice viewing SSH key fingerprints
\end{itemize}

\subsection{System Monitoring and Logs}

Practice monitoring system processes and understanding log files.

Monitor system processes:

\begin{lstlisting}[style=bashstyle]
# View running processes
$ ps aux | head -20
$ ps aux | grep ssh

# Monitor system resources
$ free -h
$ df -h

# Check network connections
$ netstat -tuln | grep :22
$ ss -tuln | grep :22
\end{lstlisting}

Analyze system logs:

\begin{lstlisting}[style=bashstyle]
# Check authentication logs (if accessible)
$ sudo tail -20 /var/log/auth.log
$ sudo grep "Failed password" /var/log/auth.log

# Check system logs
$ sudo journalctl -u ssh
$ sudo journalctl --since "1 hour ago"
\end{lstlisting}

\textbf{Exercise}: Practice system monitoring:
\begin{itemize}
    \item Use \cmd{ps} to view running processes
    \item Use \cmd{free} and \cmd{df} to monitor system resources
    \item Use \cmd{netstat} and \cmd{ss} to check network connections
    \item Use \cmd{journalctl} to view system logs
    \item Understand the importance of log monitoring for security
\end{itemize}

\subsection{Security Hardening}

Practice implementing basic security measures and understanding security concepts.

Check system security status:

\begin{lstlisting}[style=bashstyle]
# Check current firewall status
$ sudo ufw status
$ sudo iptables -L

# Check system information
$ uname -a
$ whoami
$ id
\end{lstlisting}

Practice with file permissions:

\begin{lstlisting}[style=bashstyle]
# Create test files with different permissions
$ touch test-file.txt
$ chmod 755 test-file.txt
$ ls -l test-file.txt

# Practice with directory permissions
$ mkdir test-dir
$ chmod 700 test-dir
$ ls -ld test-dir
\end{lstlisting}

\textbf{Exercise}: Practice security hardening concepts:
\begin{itemize}
    \item Use \cmd{ufw} to check firewall status
    \item Use \cmd{iptables} to view firewall rules
    \item Use \cmd{chmod} to set secure file permissions
    \item Understand the principle of least privilege
    \item Practice with different permission combinations for security
\end{itemize}

\subsection{Network Security}

Practice network security concepts and tools.

Analyze network connections:

\begin{lstlisting}[style=bashstyle]
# Monitor network connections
$ netstat -an | grep :22
$ ss -tuln | grep :22

# Check for listening services
$ sudo netstat -tlnp
$ sudo ss -tlnp
\end{lstlisting}

Test network connectivity:

\begin{lstlisting}[style=bashstyle]
# Test port connectivity
$ telnet localhost 22
$ nc -zv localhost 22
\end{lstlisting}

\textbf{Exercise}: Practice network security concepts:
\begin{itemize}
    \item Use \cmd{netstat} and \cmd{ss} to monitor network connections
    \item Use \cmd{telnet} and \cmd{nc} to test network connectivity
    \item Understand the importance of monitoring network services
    \item Practice identifying listening ports and services
    \item Understand basic network security principles
\end{itemize}

\subsection{Bonus: Security Concepts}

As a bonus exercise, practice the fundamental security concepts covered in the chapter.

\begin{lstlisting}[style=bashstyle]
# Practice with system information
$ uname -a
$ cat /etc/os-release
$ whoami
$ id
$ groups

# Practice with file security
$ ls -la /etc/passwd
$ ls -la /etc/shadow
$ ls -la /bin/ls
\end{lstlisting}

\textbf{Bonus Exercise}: Demonstrate understanding of security concepts:
\begin{itemize}
    \item Use \cmd{uname} to get system information
    \item Use \cmd{whoami} and \cmd{id} to understand user identity
    \item Use \cmd{groups} to see user group memberships
    \item Understand the importance of file permissions for system security
    \item Practice identifying security-sensitive files and their permissions
\end{itemize}
