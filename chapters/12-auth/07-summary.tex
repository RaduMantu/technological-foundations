\section{Summary}
\label{sec:sec:summary}

A system is secure if it is used as it was designed.
Since it is quasi-impossible to cover all cases in which a system can be used, we cannot say that a system is perfectly secure.
We say that security is a process, not an end goal.

An attacker can seek to control a system, steal information, or abuse resources.
A defender seeks to prevent the existence of attacks or react as quickly as possible to limit the damages of an attack.
A defender will seek to ensure data confidentiality, data integrity, service availability, and privacy protection (\textit{privacy}).

For protecting a system, there are security principles that users and organizations should follow: least privilege, defense in depth, separation of mechanism from policy, modular design.

A user or administrator will be concerned with access security (who has access to the system and to what resources), transfer security (how information is transmitted on the Internet), and application security (an application to be protected from attacks or, if attacked, not to affect other applications).

There are protocols, algorithms, technologies, and applications for increasing a system's security.
A user/administrator will be continuously informed, will allocate resources, and will make constant effort to guarantee an optimal level of security. 