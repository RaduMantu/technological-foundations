\newpage

\section{Practical exercises}

\subsection{File Operations Mastery}

In this exercise, you will practice essential file and directory operations using the commands covered in the chapter.

First, create a working directory and some test files:

\begin{lstlisting}[style=bashstyle]
$ mkdir -p ~/file-ops-lab/{source,backup,archive}
$ cd ~/file-ops-lab
$ touch source/{file1.txt,file2.txt,file3.txt}
$ echo "Hello World" > source/file1.txt
$ echo "This is file 2" > source/file2.txt
$ echo "File 3 content" > source/file3.txt
\end{lstlisting}

Practice basic file operations:

\begin{lstlisting}[style=bashstyle]
# Copy files
$ cp source/file1.txt backup/
$ cp -r source/ backup/source-copy/

# Move and rename files
$ mv source/file2.txt backup/file2-renamed.txt
$ mv source/file3.txt archive/

# Create symbolic and hard links
$ ln -s source/file1.txt source-link
$ ln source/file1.txt source-hard-link

# Verify links
$ ls -la source*
$ readlink source-link
\end{lstlisting}

\textbf{Exercise}: Practice file operations:
\begin{itemize}
    \item Use \cmd{cp} to copy files and directories
    \item Use \cmd{mv} to move and rename files
    \item Create symbolic links with \cmd{ln -s}
    \item Create hard links with \cmd{ln}
    \item Use \cmd{readlink} to see where symbolic links point
\end{itemize}

\subsection{File Searching and Filtering}

Practice finding files and content using the search tools covered in the chapter.

Create a directory structure for searching:

\begin{lstlisting}[style=bashstyle]
$ mkdir -p ~/search-lab/{docs,code,data,logs}
$ cd ~/search-lab

# Create various file types
$ echo "Python script content" > code/script.py
$ echo "Configuration file" > code/config.conf
$ echo "Log entry: ERROR 404" > logs/app.log
$ echo "Another log: INFO startup" >> logs/app.log
$ echo "Data file content" > data/data.txt
$ echo "Document content" > docs/readme.txt
\end{lstlisting}

Practice different search techniques:

\begin{lstlisting}[style=bashstyle]
# Find files by name
$ find . -name "*.py"
$ find . -name "*.log"

# Find files by type
$ find . -type f
$ find . -type d

# Find files by size
$ find . -size +1k
$ find . -size -100c

# Find files by modification time
$ find . -mtime -1
$ find . -newer code/script.py

# Search file contents
$ grep -r "ERROR" .
$ grep -r "Python" .
$ grep -l "content" *
\end{lstlisting}

\textbf{Exercise}: Practice file searching:
\begin{itemize}
    \item Use \cmd{find} to locate files by name, type, size, and date
    \item Use \cmd{grep} to search for text within files
    \item Use \cmd{locate} to quickly find files (if available)
    \item Use \cmd{whereis} and \cmd{which} to find command locations
    \item Combine search commands with other tools
\end{itemize}

\subsection{Archiving and Compression}

Practice working with different archive formats and compression methods.

Create files to archive:

\begin{lstlisting}[style=bashstyle]
$ mkdir -p ~/archive-lab/{project1,project2}
$ cd ~/archive-lab

# Create sample project files
$ echo "Source code" > project1/main.c
$ echo "Header file" > project1/header.h
$ echo "Makefile" > project1/Makefile
$ echo "Documentation" > project1/README.txt

$ echo "Python script" > project2/script.py
$ echo "Requirements" > project2/requirements.txt
$ echo "Data file" > project2/data.csv
\end{lstlisting}

Practice different archiving methods:

\begin{lstlisting}[style=bashstyle]
# Create tar archives
$ tar -cvf project1.tar project1/
$ tar -czvf project1.tar.gz project1/
$ tar -cjvf project1.tar.bz2 project1/

# List archive contents
$ tar -tvf project1.tar
$ tar -tzvf project1.tar.gz

# Extract archives
$ mkdir extracted
$ tar -xvf project1.tar -C extracted/
$ tar -xzvf project1.tar.gz -C extracted/

# Work with zip archives
$ zip -r project2.zip project2/
$ unzip -l project2.zip
$ unzip project2.zip -d extracted/
\end{lstlisting}

\textbf{Exercise}: Practice archiving and compression:
\begin{itemize}
    \item Use \cmd{tar} to create and extract archives
    \item Use different compression methods (gzip, bzip2)
    \item Use \cmd{zip} and \cmd{unzip} for ZIP archives
    \item List archive contents before extracting
    \item Compare file sizes before and after compression
\end{itemize}

\subsection{I/O Redirection and Pipes}

Practice input/output redirection and command chaining with pipes.

Create test files for redirection exercises:

\begin{lstlisting}[style=bashstyle]
$ mkdir ~/io-lab
$ cd ~/io-lab

# Create sample data files
$ echo -e "apple\nbanana\ncherry\ndate\nelderberry" > fruits.txt
$ echo -e "1\n5\n3\n9\n2\n8\n4\n7\n6" > numbers.txt
$ echo -e "error: file not found\ninfo: system started\nwarning: low memory\nerror: connection failed" > log.txt
\end{lstlisting}

Practice redirection and piping:

\begin{lstlisting}[style=bashstyle]
# Output redirection
$ ls -la > file-list.txt
$ echo "New content" >> file-list.txt
$ ls -la 2> errors.txt

# Input redirection
$ sort < numbers.txt
$ wc -l < fruits.txt

# Pipes
$ cat fruits.txt | sort | uniq
$ cat log.txt | grep "error" | wc -l

# Complex pipelines
$ find . -name "*.txt" | xargs wc -l | sort -nr
$ cat numbers.txt | sort -n | head -5
$ cat log.txt | grep "error" | cut -d: -f2 | sort | uniq
\end{lstlisting}

\textbf{Exercise}: Practice I/O redirection and pipes:
\begin{itemize}
    \item Use \cmd{>} to redirect output to files
    \item Use \cmd{>>} to append output to files
    \item Use \cmd{<} to redirect input from files
    \item Use \cmd{|} to pipe output between commands
    \item Combine multiple commands with pipes
\end{itemize}

\subsection{File System Types and Data Integrity}

Practice working with different file system types and data integrity concepts.

\begin{lstlisting}[style=bashstyle]
# Check file system types
$ df -T
$ mount | grep -E "(ext4|ntfs|fat32)"

# Check data integrity with checksums
$ echo "test content" > test-file.txt
$ md5sum test-file.txt
$ sha256sum test-file.txt

# Compare files
$ cp test-file.txt test-file-copy.txt
$ diff test-file.txt test-file-copy.txt
$ cmp test-file.txt test-file-copy.txt
\end{lstlisting}

\textbf{Exercise}: Practice file system concepts:
\begin{itemize}
    \item Use \cmd{df -T} to identify file system types
    \item Use \cmd{md5sum} and \cmd{sha256sum} to verify file integrity
    \item Use \cmd{diff} and \cmd{cmp} to compare files
    \item Understand the difference between different file system types
    \item Practice backup methods using \cmd{dd} and \cmd{rsync}
\end{itemize}

\subsection{Bonus: File Management Concepts}

As a bonus exercise, practice the fundamental file management concepts covered in the chapter.

\begin{lstlisting}[style=bashstyle]
# Practice with file permissions
$ touch test-file.txt
$ chmod 755 test-file.txt
$ ls -l test-file.txt

# Practice with file attributes
$ ls -la /bin/ls
$ stat /bin/ls

# Practice with different file types
$ file /bin/ls /etc/passwd /dev/null
\end{lstlisting}

\textbf{Bonus Exercise}: Demonstrate understanding of file management concepts:
\begin{itemize}
    \item Use \cmd{chmod} to change file permissions
    \item Use \cmd{ls -la} to view file attributes
    \item Use \cmd{stat} to see detailed file information
    \item Use \cmd{file} to identify file types
    \item Understand the relationship between file operations and system security
\end{itemize}
