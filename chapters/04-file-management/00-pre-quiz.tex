\section{Pre-lecture Quiz}
\label{sec:file-mgmt:pre-quiz}

\subsection{Quiz Questions}

\textbf{Question 1:} What is the primary purpose of the `pwd` command in Unix-like systems?

\begin{enumerate}
    \item[A)] To change the current directory
    \item[B)] \textbf{To display the absolute path of the current working directory}
    \item[C)] To list files in the current directory
    \item[D)] To create a new directory
\end{enumerate}

\textbf{Question 2:} Which command is used to change directories in a file system?

\begin{enumerate}
    \item[A)] \textbf{`cd`}
    \item[B)] `pwd`
    \item[C)] `ls`
    \item[D)] `mkdir`
\end{enumerate}

\textbf{Question 3:} What does the `ls -l` command display that the basic `ls` command does not?

\begin{enumerate}
    \item[A)] More file names
    \item[B)] \textbf{Detailed information including permissions, ownership, size, and modification date}
    \item[C)] Hidden files only
    \item[D)] Directory names only
\end{enumerate}

\textbf{Question 4:} What is the purpose of the `cp` command?

\begin{enumerate}
    \item[A)] To change file permissions
    \item[B)] \textbf{To copy files and directories}
    \item[C)] To create new files
    \item[D)] To delete files
\end{enumerate}

\textbf{Question 5:} In Unix-like systems, what does the `>` operator do in command line operations?

\begin{enumerate}
    \item[A)] It compares two files
    \item[B)] \textbf{It redirects output to a file, overwriting the file if it exists}
    \item[C)] It displays file contents
    \item[D)] It creates a symbolic link
\end{enumerate}

\textbf{Question 6:} What is the main difference between `rm` and `rmdir` commands?

\begin{enumerate}
    \item[A)] `rm` is faster than `rmdir`
    \item[B)] \textbf{`rm` removes files, while `rmdir` removes empty directories only}
    \item[C)] `rm` works on Windows, `rmdir` works on Unix
    \item[D)] There is no difference between them
\end{enumerate}

\subsection{Answer Key}

\begin{enumerate}
    \item \textbf{B} - The `pwd` (print working directory) command displays the absolute path of the current working directory, helping users understand their location in the file system hierarchy.
    
    \item \textbf{A} - The `cd` (change directory) command is used to navigate between directories in the file system, allowing users to move to different locations.
    
    \item \textbf{B} - The `ls -l` command provides a long listing format that includes detailed information such as file permissions, ownership, size, and modification date, which is not shown by the basic `ls` command.
    
    \item \textbf{B} - The `cp` (copy) command is used to create copies of files and directories, allowing users to duplicate content while preserving the original.
    
    \item \textbf{B} - The `>` operator redirects the standard output of a command to a file, overwriting the file if it already exists. This is a fundamental concept in shell redirection.
    
    \item \textbf{B} - The `rm` command removes files, while `rmdir` specifically removes empty directories. For non-empty directories, `rm -r` is typically used.
\end{enumerate}
