\newpage

\section{Practical exercises}

The following tasks are styled after Capture The Flag (CTF) challenges. These
types of competitions are normally a good method of building up your pratical
skill set. For this introductory module we've kept the difficulty to a minimum.
Try to familiarize yourself with the essential Linux utilities.

\subsection{Where's my decoder ring?}

You've intercepted a transmission from a \textbf{base} in Area \textbf{64}.
It's been saved in \texttt{01-transmission.txt}. Try to decode it.

\subsection{State of the art encryption}

You've found a zip archive but it's contents have been encrypted. If you
try to list its contents with \texttt{unzip} you will find the password. But
this too was encrypted with a \textit{particular instance} of Caesar's cipher.

\subsection{Just try and run it}

The flag this time is stored in a JPEG image. Or is it a JPEG? \\
Hint: Yes, it's a JPEG. Don't execute weird stuff, even if \texttt{file} says
      it's a script.

\subsection{Substitution cipher}

Someone used a substitution cipher to hide the flag from us. Luckily, these
types of cyphers can be broken iteratively. Look for bigrams, trigrams or try
guessing words within the text to deduce the missing substitutions.

Use the helper script we've provided. Complete the dictionary with your
substitutions. By running the script, the substituted letters will be
highlighted in red. Meanwhile, the remaning letters will be printed normally.

Hint: The cipher seems to only cover letters, not digits. Seems like there's a
      date in there somewhere. What month could that be?

