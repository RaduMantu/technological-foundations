\section{Overview}

What is cybersecurity? Vague questions usually have many answers that are
correct in their own respect. In this module, the answer that we are going to
work with and build upon as we progress is as follows: \textit{"Cybersecurity
is, given an \textbf{attacker's model} and a specific \textbf{context}, the
technique to control \textbf{who} may \textbf{use} or \textbf{modify} the
\textbf{data}"}. Let's break this down.

\subsection{Attacker model}

First, what is an \textbf{attacker model}? The attacker model is a conceptual
framework that allows us to describe potential adversaries based on our
assumptions in regards to their goals, resources, knowledge, and methods. Some
key aspects to consider when formulating the attacker model are as follows:

\begin{itemize}
    \item \textbf{Capabilities:} The computational power at the attacker's
          disposal, his access to specialized tools, or insider privileges
          (e.g., physical access to our assets, etc.)
    \item \textbf{Knowledge:} What the attacker knows. This can include our
          system architecture (including defensive measures in place), source
          code of deployed services, partial data, etc.
    \item \textbf{Motivation:} Knowing the objective can help predict the
          behaviour of the attacker. For example, attempts at causing disruption
          (i.e., denial of service) will be more overt than attempts at
          infiltration with the purpose of data theft.
    \item \textbf{Methods:} These are techniques that the attacker may employ to
          reach his objective. These include but are not limited to social
          engineering, brute force attacks, exploiting known or zero-day
          software vulnerabilities, etc.
\end{itemize}

Note that these are very general aspects that you may need to consider. However,
each specific situation will demand additional consideration. For example, in
cryptography one must determine whether the attacker is assumed to be
\textbf{passive} or \textbf{active}. A passive attacker can only eavesdrop on
communication, while an active attacker can intercept and modify in-transit
data. Some algorithms (e.g., Diffie-Hellman key exchange
\cite{mitra2021prevention}) are resistant to passive attackers but provide no
guarantees against their stronger counterpart.

\subsection{Security context}

The \textbf{security context} refers to the specific environment, conditions
and constraints that shape the assumptions and parameters of the attacker model.
In other words, the security context ensures that the model is tailored to the
system's unique characteristics and operational environment. Below we enumerate
a few key elements that are typically considered:

\begin{itemize}
    \item \textbf{System architecture:} The structure of the system, including
          both hardware and software components, as well as data flows. This
          determines what parts of the system are exposed to the attacker and
          allows the security team to focus on relevant threats. For example,
          when securing a web application, one can assume that the attacker has
          access to the public endpoints but not to the internal server
          configuration. As a result, the focus should be on external threats
          such as SQL injections attacks \cite{clarke2009sql} and DDoS attacks
          \cite{ramesh2023hybrid}.

    \item \textbf{Assets and data sensitivity:} The critical resources or data
          the system handles. Understanding what these are can help identify the
          attacker's likely targets and motivations. For example, the security
          context of a banking system should prioritize the safety of the user's
          financial data. As a result, the attacker model should include
          Advanced Persistent Threats (APT) \cite{alshamrani2019survey} that
          target transaction systems. Note that depending on the sensitivity of
          the data in question, regulations may impose restrictions on what
          defensive solutions may be used.

    \item \textbf{Access points and boundaries:} Entry points to the system
          (e.g., APIs, user interfaces, physical devices) and trust boundaries
          (e.g., between users, servers, first-party and third-party services).
          This information helps narrow down where an attacker might gain entry
          or escalate privileges, shaping assumptions about their starting
          position. For example, can a corporate computer network be compromised
          from within due to lax Bring Your Own Device (BYOD) policies
          \cite{alotaibi2018review} or are the threats purely external? What if
          the attacker gains access to VPN credentials of an employee and thus
          gains access to the internal network?
\end{itemize}
