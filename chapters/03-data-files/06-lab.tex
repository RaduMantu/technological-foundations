\newpage

\section{Practical exercises}

\subsection{File System Navigation}

In this exercise, you will practice navigating the file system hierarchy and understanding the difference between absolute and relative paths.

First, create a simple directory structure to work with:

\begin{lstlisting}[style=bashstyle]
$ mkdir -p ~/file-system-lab/{documents,media,projects}
$ cd ~/file-system-lab
$ touch documents/report.txt media/photo.jpg projects/script.sh
\end{lstlisting}

Now practice using different path types:

\begin{lstlisting}[style=bashstyle]
# Display your current location
$ pwd

# Navigate using relative paths
$ cd documents
$ pwd
$ cd ../media
$ pwd

# Navigate using absolute paths
$ cd /home/student/file-system-lab/projects
$ pwd

# Practice with . and .. references
$ cd ../../
$ pwd
$ ls -la
\end{lstlisting}

\textbf{Exercise}: Practice the following navigation tasks:
\begin{itemize}
    \item Use \cmd{pwd} to show your current location
    \item Use \cmd{cd} with relative paths to navigate between directories
    \item Use \cmd{cd} with absolute paths to jump to specific locations
    \item Use \cmd{cd ..} to go up one directory level
    \item Use \cmd{cd ~} to return to your home directory
\end{itemize}

\subsection{File Type Identification}

The \cmd{file} command is essential for determining the actual type of files, regardless of their extensions.

Create some test files with misleading extensions:

\begin{lstlisting}[style=bashstyle]
# Create a text file but give it a .jpg extension
$ echo "This is actually a text file" > fake-image.jpg

# Create a binary file (executable) but give it a .txt extension
$ cp /bin/ls fake-script.txt

# Create a directory but give it a .file extension
$ mkdir fake-directory.file
\end{lstlisting}

Now use the \cmd{file} command to identify the actual types:

\begin{lstlisting}[style=bashstyle]
$ file fake-image.jpg
$ file fake-script.txt
$ file fake-directory.file
$ file /bin/ls
$ file /etc/passwd
\end{lstlisting}

\textbf{Exercise}: Practice identifying file types:
\begin{itemize}
    \item Use \cmd{file} to check the type of files with misleading extensions
    \item Compare the file extension with the actual file type
    \item Identify different types of files in your system directories
    \item Understand the difference between text and binary files
\end{itemize}

\subsection{Path Construction and Navigation}

Practice building and navigating paths using the concepts learned in the chapter.

\begin{lstlisting}[style=bashstyle]
# Start from your home directory
$ cd ~

# Create a symbolic link
$ ln -s /usr/bin/python3 my-python

# Practice path resolution
$ ls -l my-python
$ readlink my-python

# Work with paths containing spaces
$ mkdir "directory with spaces"
$ cd "directory with spaces"
$ touch "file with spaces.txt"
$ ls -la
\end{lstlisting}

\textbf{Exercise}: Practice path construction:
\begin{itemize}
    \item Create symbolic links and understand how they work
    \item Work with files and directories that have spaces in their names
    \item Practice using \cmd{readlink} to see where symbolic links point
    \item Understand the difference between symbolic links and regular files
\end{itemize}

\subsection{File System Hierarchy Understanding}

Explore the standard Linux file system hierarchy and understand the purpose of each directory.

\begin{lstlisting}[style=bashstyle]
# Explore the root directory
$ ls -la /

# Examine system directories
$ ls /bin | head -10
$ ls /usr/bin | head -10
$ ls /etc | head -10

# Check your home directory structure
$ ls -la ~
$ echo $HOME
\end{lstlisting}

\textbf{Exercise}: Explore the file system hierarchy:
\begin{itemize}
    \item List the contents of the root directory (\file{/})
    \item Examine the contents of \file{/bin}, \file{/usr/bin}, and \file{/etc}
    \item Identify the purpose of different system directories
    \item Understand the difference between \file{/bin} and \file{/usr/bin}
    \item Find examples of hidden files (starting with \file{.})
\end{itemize}

\subsection{Bonus: File System Concepts}

As a bonus exercise, practice the fundamental concepts covered in the chapter.

\begin{lstlisting}[style=bashstyle]
# Practice with file attributes
$ ls -la /bin/ls
$ ls -la /etc/passwd

# Understand file metadata
$ stat /bin/ls
$ stat /etc/passwd

# Practice with different file types
$ file /bin/ls /etc/passwd /dev/null
\end{lstlisting}

\textbf{Bonus Exercise}: Demonstrate understanding of file system concepts:
\begin{itemize}
    \item Use \cmd{ls -la} to view file attributes and permissions
    \item Use \cmd{stat} to see detailed file metadata
    \item Identify different file types using \cmd{file}
    \item Understand the relationship between file extensions and actual file types
\end{itemize}
