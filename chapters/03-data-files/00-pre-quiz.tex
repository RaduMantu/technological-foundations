\section{Pre-lecture Quiz}
\label{sec:data-files:pre-quiz}

\subsection{Quiz Questions}

\textbf{Question 1:} What is the primary purpose of a file system in an operating system?

\begin{enumerate}
    \item[A)] To provide internet connectivity
    \item[B)] \textbf{To organize and manage files and directories in a hierarchical structure}
    \item[C)] To control user access to the system
    \item[D)] To manage system memory allocation
\end{enumerate}

\textbf{Question 2:} In a file system hierarchy, what does the root directory represent?

\begin{enumerate}
    \item[A)] The user's home directory
    \item[B)] The directory containing system files only
    \item[C)] \textbf{The top-level directory that contains all other directories and files}
    \item[D)] A directory that cannot be accessed by users
\end{enumerate}

\textbf{Question 3:} What is the difference between an absolute path and a relative path?

\begin{enumerate}
    \item[A)] An absolute path is longer than a relative path
    \item[B)] \textbf{An absolute path starts from the root directory, while a relative path starts from the current directory}
    \item[C)] An absolute path can only be used by administrators
    \item[D)] There is no difference between them
\end{enumerate}

\textbf{Question 4:} What does the file extension typically indicate?

\begin{enumerate}
    \item[A)] The file's creation date
    \item[B)] The file's size in bytes
    \item[C)] \textbf{The type or format of the file}
    \item[D)] The file's access permissions
\end{enumerate}

\textbf{Question 5:} In Unix-like systems, what do the special directory references "." and ".." represent?

\begin{enumerate}
    \item[A)] Hidden files that cannot be seen
    \item[B)] \textbf{"." represents the current directory and ".." represents the parent directory}
    \item[C)] System files that should not be modified
    \item[D)] Temporary files created by the system
\end{enumerate}

\subsection{Answer Key}

\begin{enumerate}
    \item \textbf{B} - The file system is responsible for organizing and managing files and directories in a hierarchical structure, providing a logical way to store, organize, and access data on storage devices.
    
    \item \textbf{C} - The root directory (/) is the top-level directory in the file system hierarchy that contains all other directories and files. It serves as the starting point for all absolute paths.
    
    \item \textbf{B} - An absolute path provides the complete path from the root directory to the target file or directory, while a relative path provides the path relative to the current working directory.
    
    \item \textbf{C} - File extensions typically indicate the type or format of the file, helping both users and the operating system understand how to handle the file (e.g., .txt for text files, .jpg for images).
    
    \item \textbf{B} - In Unix-like systems, "." is a special reference to the current directory, and ".." is a special reference to the parent directory. These are used for navigation and relative path construction.
\end{enumerate}
