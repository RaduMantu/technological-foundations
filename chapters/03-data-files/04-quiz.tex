\section{Post-lecture Quiz}
\label{sec:data-files:post-quiz}

\subsection{Quiz Questions}

\textbf{Question 1:} What is the main advantage of using a hierarchical file system organization?

\begin{enumerate}
    \item[A)] It reduces the amount of storage space needed
    \item[B)] \textbf{It allows for efficient organization and quick access to large numbers of files}
    \item[C)] It automatically compresses files to save space
    \item[D)] It prevents users from accessing system files
\end{enumerate}

\textbf{Question 2:} When using the `file` command in Linux, what information does it provide?

\begin{enumerate}
    \item[A)] The file's creation date and time
    \item[B)] The file's size and permissions
    \item[C)] \textbf{The file's type and format based on its content, not just its extension}
    \item[D)] The file's owner and group information
\end{enumerate}

\textbf{Question 3:} In the context of file attributes, what does the `ls -l` command display?

\begin{enumerate}
    \item[A)] Only the file names in the current directory
    \item[B)] \textbf{Detailed information including permissions, ownership, size, and modification date}
    \item[C)] Only hidden files in the directory
    \item[D)] The total number of files in the directory
\end{enumerate}

\textbf{Question 4:} What is the primary difference between text files and binary files?

\begin{enumerate}
    \item[A)] Text files are always smaller than binary files
    \item[B)] \textbf{Text files contain human-readable characters, while binary files contain data in a format that requires specific programs to interpret}
    \item[C)] Text files can only be created by text editors
    \item[D)] Binary files cannot be copied or moved
\end{enumerate}

\textbf{Question 5:} In a three-level hierarchical organization with 100 directories at each level, how many files can be stored if each final directory contains 100 files?

\begin{enumerate}
    \item[A)] 1,000 files
    \item[B)] 10,000 files
    \item[C)] \textbf{1,000,000 files (100 × 100 × 100)}
    \item[D)] 100,000 files
\end{enumerate}

\textbf{Question 6:} What does file metadata typically include?

\begin{enumerate}
    \item[A)] Only the file's name and size
    \item[B)] \textbf{Information about the file such as size, creation date, modification date, owner, and permissions}
    \item[C)] Only the file's content
    \item[D)] Only the file's location in the directory structure
\end{enumerate}

\textbf{Question 7:} Why is it important to understand file formats and extensions?

\begin{enumerate}
    \item[A)] To determine the file's security level
    \item[B)] \textbf{To know which applications can open and properly display the file content}
    \item[C)] To calculate the file's storage cost
    \item[D)] To determine the file's network transfer speed
\end{enumerate}

\subsection{Answer Key}

\begin{enumerate}
    \item \textbf{B} - A hierarchical file system allows for efficient organization of large numbers of files by creating a tree-like structure, making it easier to locate and access files quickly.
    
    \item \textbf{C} - The `file` command analyzes the actual content of a file to determine its type and format, which is more reliable than relying solely on file extensions that can be misleading.
    
    \item \textbf{B} - The `ls -l` command provides detailed listing information including file permissions, ownership, size, and modification date, giving users comprehensive information about files in the directory.
    
    \item \textbf{B} - Text files contain human-readable characters that can be displayed and edited with basic text editors, while binary files contain data in a specific format that requires specialized programs to interpret correctly.
    
    \item \textbf{C} - In a three-level hierarchy with 100 directories at each level and 100 files per final directory: 100 (first level) × 100 (second level) × 100 (files per directory) = 1,000,000 files.
    
    \item \textbf{B} - File metadata includes various attributes about the file such as its size, creation and modification dates, owner information, permissions, and other properties that describe the file without containing its actual content.
    
    \item \textbf{B} - Understanding file formats and extensions is crucial for knowing which applications can properly open, display, and manipulate the file content, ensuring users can work with files effectively.
\end{enumerate}
